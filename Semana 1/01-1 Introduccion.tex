\documentclass{beamer}
\usepackage[utf8x]{inputenc}
\usepackage{amsmath,amsfonts,amssymb}
\usepackage{lmodern}
% \usepackage{fixltx2e}
\usepackage{hyperref}
\usepackage{xcolor}
\usepackage{pgfplots}
\usepackage{tikz}
\usepackage{array}

\usetikzlibrary{arrows,positioning} 

\pgfplotsset{my style/.append style={axis x line=middle, axis y line=middle}}
\pgfplotsset{compat=1.9}

\DeclareMathOperator*{\argmax}{arg\,max}
\DeclareMathOperator*{\argmin}{arg\,min}
\DeclareMathOperator*{\E}{E}

\usecolortheme{shigoto}

\definecolor{darkblue}{rgb}{0.2,0.2,0.6}
\definecolor{darkred}{rgb}{0.6,0.1,0.1} 
\definecolor{darkgreen}{rgb}{0.2,0.6,0.2}


\title{Preparadurías de Mate 4}
\institute{Universidad Simón Bolívar}
\author[Contreras]{Carlos Contreras }
\date{Ene-Mar 2021}



\begin{document}

\setbeamercolor*{section in toc}{fg=black}
\setbeamercolor*{item}{fg=AwayWinter!80!black}

%%%%%%%%%%%%%%%%%%%%%%%%%%%%%%%%%%%%%%%%%%%%%%%%%%%%%%%%%%%%%%%%%%
\begin{frame}
     \thispagestyle{empty}
     \titlepage
\end{frame}
%%%%%%%%%%%%%%%%%%%%%%%%%%%%%%%%%%%%%%%%%%%%%%%%%%%%%%%%%%%%%%%%%%
%-----------------------------------------------------------------
\begin{frame}
\frametitle{¿Quién soy yo?}

\begin{itemize}
 \item Vivo en Edmonton, Canadá
 \item Carnet 03, egresado de matemáticas aplicadas
 \item PhD en matemáticas aplicadas en la Universidad de Alberta
 \item Passión: biología matemática y machine learning
 \item Actualmente investigador e instructor en la Universidad de Alberta
\end{itemize}


\end{frame}
%%%%%%%%%%%%%%%%%%%%%%%%%%%%%%%%%%%%%%%%%%%%%%%%%%%%%%%%%%%%%%%%%%
%-----------------------------------------------------------------
\begin{frame}
\frametitle{Programa}
\begin{itemize}
 \item Series y sucesiones
 \begin{itemize}
  \item Series de potencias
  \item Series de Taylor
 \end{itemize}
 \item Ecuaciones diferenciales ordinarias
 \begin{itemize}
  \item Ecuaciones de primer orden
  \item Sistemas de ecuaciones de primer orden
  \item Ecuaciones lineales de segundo orden
 \end{itemize}
\end{itemize}
\end{frame}
%%%%%%%%%%%%%%%%%%%%%%%%%%%%%%%%%%%%%%%%%%%%%%%%%%%%%%%%%%%%%%%%%%
%-----------------------------------------------------------------
\begin{frame}
\frametitle{Evaluaciones}
Dos evaluaciones de 20\% cada una en semana 4 y semana 9.

\vspace{10pt}
Formato: TBD.

\vspace{10pt}
Total: 40\% será evaluado en las prepas.
\end{frame}
%%%%%%%%%%%%%%%%%%%%%%%%%%%%%%%%%%%%%%%%%%%%%%%%%%%%%%%%%%%%%%%%%%
%-----------------------------------------------------------------
\begin{frame}
\frametitle{Series de potencia}
Representación de una función en términos de polinomios
\[f(x)=a_{0}+a_{1}x+a_{2}x^{2}+a_{3}x^{3}+\cdots \onslide<2->{=\sum_{n=0}^{\infty}a_{n}x^{n}}\]

Por ejemplo,
\begin{itemize}
 \item $e^{x} = 1+x+\frac{1}{2}x^{2}+\frac{1}{6}x^{3}+\cdots = \onslide<2->{\sum_{n=0}^{\infty}\frac{1}{n!}x^{n}}.$
 \item $\sin(x) = x-\frac{1}{6}x^{3}+\frac{1}{120}x^{5}+\cdots = \onslide<2->{\sum_{n=0}^{\infty}\frac{(-1)^{n}}{(2n+1)!}x^{2n+1}}.$
 \item $x(x-2)(x+2)=0-4x+0x^{2}+x^{3}+0x^{4}+\cdots$
\end{itemize}

\vspace{10pt}
\onslide<3>{
Representación alrededor de un punto
\[f(x)=a_{0}+a_{1}(x-a)+a_{2}(x-a)^{2}+\cdots =\sum_{n=0}^{\infty}a_{n}(x-a)^{n}\]
}
\end{frame}
%%%%%%%%%%%%%%%%%%%%%%%%%%%%%%%%%%%%%%%%%%%%%%%%%%%%%%%%%%%%%%%%%%
%-----------------------------------------------------------------
\begin{frame}
\frametitle{Series de Taylor}
Los coefficientes de las serie
\[f(x)=a_{0}+a_{1}(x-a)+a_{2}(x-a)^{2}+\cdots =\sum_{n=0}^{\infty}a_{n}(x-a)^{n}\]
son
\[a_{n}=\frac{f^{(n)}(a)}{n!}\]
Las derivadas de $f(x)$ son usadas para la representación!
\pause

\vspace{10pt}
Aproximación de una función
\[\sin(x) \approx x-\frac{1}{6}x^{3}+\frac{1}{120}x^{5}\]
\end{frame}
%%%%%%%%%%%%%%%%%%%%%%%%%%%%%%%%%%%%%%%%%%%%%%%%%%%%%%%%%%%%%%%%%%
%-----------------------------------------------------------------
\begin{frame}
\frametitle{Ecuaciones differenciales de primer orden}
Ecuación lineal
\[y=ax+b\]
Ecuación differencial lineal de primer orden
\[y'=a(x)y+b(x)\]
\pause
Ecuación differenciales lineales
\begin{itemize}
    \item Autonomas \[y'=f(y)\]
    \item No autonomas \[y'=f(x,y)\]
\end{itemize}


\end{frame}
%%%%%%%%%%%%%%%%%%%%%%%%%%%%%%%%%%%%%%%%%%%%%%%%%%%%%%%%%%%%%%%%%%
%-----------------------------------------------------------------
\begin{frame}
\frametitle{Sistemas de ecuaciones differenciales de primer orden}
Ecuación differencial lineal autonoma de primer orden
\[y'=f(y)\]
Varias ecuaciones juntas
\[
\begin{aligned}
y_{1}'&=f(y_{1},y_{2}) \\
y_{2}'&=f(y_{1},y_{2})
\end{aligned}
\]


\end{frame}
%%%%%%%%%%%%%%%%%%%%%%%%%%%%%%%%%%%%%%%%%%%%%%%%%%%%%%%%%%%%%%%%%%
%-----------------------------------------------------------------
\begin{frame}
\frametitle{Ecuaciones differenciales de segundo orden}
Ecuación differencial lineal de primer orden
\[y'+p(x)y=q(x)\]
Ecuación differencial lineal de segundo orden
\[y''+p(x)y'+q(x)y=r(x)\]
\end{frame}
%%%%%%%%%%%%%%%%%%%%%%%%%%%%%%%%%%%%%%%%%%%%%%%%%%%%%%%%%%%%%%%%%%
\end{document}
