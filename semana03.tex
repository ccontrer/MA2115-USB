\documentclass[12pt,oneside]{book}


\usepackage{remotelecture}
\usepackage{amsmath}
\usepackage{amsfonts}
\usepackage{amssymb}
\usepackage{textcomp}
\usepackage[margin=60pt]{geometry}
\usepackage{fancyhdr}
\usepackage{titlesec}
\usepackage{everyshi}% http://ctan.org/pkg/everyshi
\usepackage{lipsum}
\usepackage{enumerate}
\usepackage{tabularx}
\usepackage{graphicx}


% definitions
\newcounter{weekpage}

% everyshi options
\EveryShipout{\stepcounter{weekpage}}% Step pagecntr every page



% titlesec options
\titleformat{\chapter}[display]
  {\normalfont\bfseries}{\Large MA-2115: Matemáticas 4}{10pt}{\huge}


% fancyhrd options
\setlength{\headheight}{15pt}
\pagestyle{fancy}

\renewcommand{\chaptermark}[1]{ \markboth{#1}{} }
\renewcommand{\sectionmark}[1]{ \markright{#1}{} }

\fancypagestyle{plain}{%
    \fancyhf{}
    \fancyfoot[C]{\thepage}
    \renewcommand{\headrulewidth}{0pt}
}

\fancypagestyle{plainweek}{%
    \fancyhf{}
%     \fancyfoot[C]{\bf \thepage~(\theweekpage)}
    \fancyfoot[C]{\bf \theweekpage}
    \renewcommand{\headrulewidth}{0pt}
}

\lhead{\bf Mate 4}
\chead{{\bf\leftmark}}
\rhead{\rightmark}
% \cfoot{\bf\thepage~(\theweekpage)}
\cfoot{\bf\theweekpage}


% Title
\title{\bf MA-2115: Matemáticas 4}
\author{Carlos Contreras}
\date{Universidad Simón Bolívar \\ Ene-Mar 2021}





\begin{document}


% \thispagestyle{empty}

% \frontmatter

% \maketitle

% \tableofcontents


% \chapter{Preface}


% \mainmatter


\chapter{Semana 3}
\setcounter{weekpage}{1}
\thispagestyle{plainweek}

\setcounter{chapter}{3}

\section{Criterios de convergencia}

\begin{enumerate}


 \item Criterio de comparación
 
  \vspace{100pt}
 
 
 \item Criterio de comparación usando límite

   \vspace{100pt}
 
 
 \item Criterio de la integral
 
   \vspace{100pt}
   
   
 \item Criterio de la serie $p$
 
 
 \newpage
 
 \item \textit{Ejercicio 3.1}
 
 Determine la convergencia de $\displaystyle \sum_{n=1}^{\infty}e^{-n}$.
 
 \vspace{180pt}
 
 
 
 \item \textit{Ejercicio 3.2}
 
 Determine la convergencia de $\displaystyle \sum_{n=1}^{\infty}\frac{1}{\sqrt[3]{3n^3+1}}$.
 
 
 \vspace{180pt}
 
 \item \textit{Ejercicio 3.3}
 
 Determine la convergencia de $\displaystyle \sum_{n=1}^{\infty}\frac{1+\sin(n)}{\sqrt[p]{n}}$.
 
 
 \newpage
 
 \vspace{220pt}
 
 

 \item Criterio del cociente
 
  \vspace{100pt}
 
 
 \item Criterio de la raíz

   \vspace{120pt}
   
   
 
 \item \textit{Ejercicio 3.3}
 
 Determine la convergencia de $\displaystyle \sum_{n=1}^{\infty}\frac{(n+1)!}{n^n}$.

 
 
   \vspace{120pt}
   
   
 
 \item \textit{Ejercicio 3.4}
 
 Determine los valores de $a>0$ para los cuales la serie $\displaystyle \sum_{n=1}^{\infty} \left( a + \frac{1}{n}\right)^{n}$ converge.

 
\end{enumerate}

\newpage

\section{Series alternantes}

\begin{enumerate}


 \item \textit{Definición}
 
  \vspace{100pt}
  
  
  \item Criterio de series alternantes (criterio de Leibnitz)
  
 
  \vspace{100pt}
  
  
  \item \textit{Ejercicio 3.5}
  
    Determine la convergencia de $\displaystyle \sum_{n=1}^{\infty} \frac{\cos(n\pi)}{n\ln^2(n)} $.


\end{enumerate}
    
\newpage
    
\section{Convergencia absoluta}

\begin{enumerate}


 \item \textit{Definición}
 
  \vspace{100pt}
  
  
  \item Criterio de series alternantes (criterio de Leibnitz)
  
 
  \vspace{100pt}
  
  
  \item \textit{Ejercicio 3.5}
  
    Determine la convergencia absoluta de $\displaystyle \sum_{n=1}^{\infty} \frac{(-1)^{n}}{n} $.

  \vspace{100pt}
  
  
  \item \textit{Ejercicio 3.6}
  
    Determine la convergencia absoluta de $\displaystyle \sum_{n=1}^{\infty} \frac{\cos(n\pi)}{n\ln^2(n)} $.

\end{enumerate}

\end{document}



