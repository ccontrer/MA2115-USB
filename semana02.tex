\chapter{Semana 2}
\setcounter{weekpage}{1}
\thispagestyle{plainweek}


\section{Sucesiones infinitas}

\begin{enumerate}


\item Definición

\vspace{100pt}

\item Convergencia

\vspace{100pt}

\item Propiedades

\vspace{100pt}

% \item \textit{Ejercicios 1.1}: Determine si la sucesión converge o diverge. Si converge, calcule el límite.
% 
% \begin{enumerate}
%  \item \[a_{n}=(-1)^{n}\cos(n\pi)\]
%  
%   \vspace{40pt}
%   
%  \item \[a_{n}=\]
% \end{enumerate}


\newpage

\item Sucesiones monótonas

\vspace{200pt}


\item \textit{Ejercicio 1.2}

Sea $a_{ n }$ la sucesión definida por $a _{1} =2$,  $a_{n+1} = \tfrac{1}{2} (a _{n} + 4)$ para $n \geq 2$. Determine si la sucesión $a _{n} $ es convergente o no. En caso afirmativo calcule el límite.


\newpage

\item Principio de inducción: \textit{ejemplo illustrativo}

\includegraphics[width=400pt]{figures/induccion.png}


\vspace{200pt}


\item \textbf{Aplicaciones}

\begin{enumerate}
 \item \textit{Fibonacci}: $F_{n}=F_{n-1}+F_{n-2}$, para $n > 1$, y $F_{0}=F_{1}=1$.

 \vspace{40pt}


\item \textit{Concentración de fármacos}: $C_{n+1}=rC_{n}+d$, dada una concentración inicial $C_{0}$.
 
\end{enumerate}




\end{enumerate}

\newpage

\section{Series}

\begin{enumerate}
 \item Definición
 
 \vspace{120pt}
 
 
 \item Propiedades
 
 \vspace{120pt}
 
 
 \item Serie armónica: \[\sum_{n=1}^{\infty}\frac{1}{n}\]
 
 \vspace{120pt}
 
 
 \item Serie geométrica: $$\sum_{n=1}^{\infty}ar^{n-1}$$
 
 \newpage
 
 
 \item Criterio de comparación ordinaria
 
  \vspace{100pt}
 
 
 \item Criterio de comparación usando límite

   \vspace{120pt}
 
 
 \item Criterio de la integral
 
   \vspace{120pt}
 
 
 \item \textit{Ejercicio 1.3}
 
 Determine la convergencia de $\displaystyle \sum_{n=1}^{\infty}\frac{e^{-n}}{\sqrt[p]{n}}$.
 

 
\end{enumerate}




